\documentclass[a4paper,10pt]{article}
\usepackage[utf8]{inputenc}
\usepackage{amsmath}
\usepackage{graphicx}
\usepackage{geometry}
\usepackage{setspace}
\geometry{top=0.5in, bottom=0.5in, left=0.5in, right=0.5in}
\setstretch{1} 
\title{\textbf{Analysis of Annual Temperature AutoCorrelations Using Permutation Testing}}
\author{Laiyin Zhou, Yaxin Liu, Sebastian Dohne, and Yangfeng Wang}
\date{04/12/2024}
\begin{document}

\maketitle
\section*{Introduction}
This study investigates whether there is a significant correlation between consecutive years' annual mean temperatures in Key West. To achieve this, we computed the observed correlation coefficient and used a permutation test to assess its statistical significance.The data consists of annual mean temperatures. The observed correlation coefficient was calculated between the temperatures of year \( t \) and year \( t+1 \). To test the null hypothesis that the observed correlation is due to random chance, a permutation test was conducted. The temperature data was randomly shuffled 10,000 times, and a correlation coefficient was calculated for each permutation. The p-value was estimated as the proportion of permuted correlations greater than or equal to the observed correlation.
\section*{Results}
The observed correlation coefficient was: 0.649.
The approximate p-value from the permutation test was: 0.0006.
This indicates that the observed correlation is highly statistically significant. Figure \ref{fig:histogram} shows the distribution of permuted correlation coefficients, with the observed correlation marked as a red dashed line.
\begin{figure}[h!]
    \centering
    \includegraphics[width=0.5\textwidth, height=0.3\textheight]{../results/Coefficients.pdf}
    \caption{Histogram of correlation coefficients from 10,000 permutations. The red dashed line represents the observed correlation.}
    \label{fig:histogram}
\end{figure}
\section*{Conclusion}
The observed correlation coefficient suggests a moderate positive relationship between consecutive years' temperatures in Key West. The extremely low p-value (\(< 0.001\)) strongly rejects the null hypothesis, indicating that the correlation is not due to random chance. This result may reflect underlying climatic patterns or environmental inertia.
\immediate\write18{mv TAutoCorrLatexCode.pdf ../results/}
\end{document}
